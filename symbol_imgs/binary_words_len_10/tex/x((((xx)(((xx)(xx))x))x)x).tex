\documentclass[border=5pt]{standalone}
\begin{document}
$x((((xx)(((xx)(xx))x))x)x)$
\end{document}